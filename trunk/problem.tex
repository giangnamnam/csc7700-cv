\documentclass{article}
\usepackage{geometry}
\geometry{   rmargin=1in
            ,lmargin=1in
            ,tmargin=1in
            ,bmargin=1in
         }

\begin{document}

\section{Team}

\begin{itemize}
 \item Team name: Context-free
 \item Team members: Kevin Cherry, Robert Firth, Dennis Castleberry
 \item Responsibilities: 
\end{itemize}

\section{Problem Statement}

\subsection{Problem definition}

Self-driving vehicles rely heavily on computer vision algorithms for efficient
detection of signs, lane markings, pedestrians, and other vehicles.  With
respect to pedestrian crosswalk and child crossing sign detection, the
efficiency and accuracy of these algorithms is imperative to minimize
pedestrian casualties. Thus, ...

\subsection{Overview of the framework and algorithm}

\textbf{SURF} will be used to detect sign edges efficiently.  SURF (speeded-up
robust feature) is a combined detector and descriptor; the detector is based
off the Hessian matrix (used for detecting curvature of edges) and the
descriptor is based off SIFT (scale-invariant feature transform).

\section{Project Description}

\subsection{Goal}

Our proposed goal is to improve the efficiency and accuracy of 
pedestrian crosswalk and child crossing sign detection.

\subsection{Objectives}

\begin{itemize}
 \item Resolve inaccuracies and ineffiencies in existing sign detection implementations
 \item Broaden the scope of detected objects which could minimize civilian casualties
\end{itemize}

\subsection{Tasks}

We will first run existing sign detection implementations against our data set
to isolate inaccuracies and inefficiencies in them.  Using our observations
from these runs, we will incrementally improve the implementations to resolve
said inaccuracies and inefficiencies and broaden the scope of the detected
objects. Finally, we will run the original and improved implementations
against a new data set and compare the differences in accuracy and efficiency. 

\subsection{Data}

We will capture video from areas dense in pedestrian crosswalk and child
crossing signs in a variety of weather conditions (normal weather, nighttime,
light and heavy rain); then segment the video to isolate the target objects.
We require two data sets; one to explore existing implementations and make
incremental improvements, and a second to test the improved implementation.

\subsection{Resources}

Because of the need for real-time detection, the detection system must not be
computationally expensive. Therefore, laptop computers will be used to test the
feasibility of the system (~1.4 GHz dual-core with ~4 GB memory).

\end{document}
