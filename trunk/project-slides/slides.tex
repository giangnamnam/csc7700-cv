\documentclass{beamer}
\usepackage{amsmath}
\usetheme{Copenhagen}
\usecolortheme[named=red]{structure}

\author{Castleberry, Cherry, and Firth}
\title{Efficient Stop \& Warning Sign and Pedestrian Detection}
\date{\today}

\begin{document}

\maketitle

\section{Introduction and Overview}
\frame{ \frametitle{Overview}
 \begin{itemize}
   \item We implemented the following:
   \begin{itemize}
    \item Stop-sign detector
    \item Warning sign detector
    \item Pedestrian detector
   \end{itemize} 
 \end{itemize} 
}

\section{Stop Signs}
\frame{ \frametitle{Stop Sign Detection, SURF}
 \begin{itemize} 
  \item We created a wrapper to the built-in EmguCV SURF detector
        to use as a basis of comparison for our own method.
  \item 
 \end{itemize} 
}

\frame{ \frametitle{Stop Sign Detection, Integral Images}
 \begin{itemize} 
  \item We developed a method for detecting stop signs based upon
        the use of integral images which we encountered in the
        SURF algorithm.
  \item To summarize the method, we use integral images from both
        the left-hand side (top left) and the right-hand side
        (bottom right). Then we consider only the LHS and RHS
        integral images along the diagonal of the image.  We
        difference these, then fit a Gaussian curve to the
        resulting vector.  We threshold the curve at one standard
        deviation to form a bounding box for the stop sign.
 \end{itemize} 
}

\frame{ \frametitle{Stop Sign Detection, Integral Images}
 \begin{itemize} 
  \item To begin our method, we scale the $N \times M$ image
        to $N \times N$ for $N<M$, $M \times M$ for $M<N$.  We
        require a square matrix to extract a particular vector.
  \item Recall that the formula for computing the integral image
        $\mathcal{I_{-}}$ at a pixel $(x, y)$ with intensity value
        $I(x,y)$ is:
  \begin{equation}
   \mathcal{I}_{-}(x,y) = \sum_{i=0}^{n_x} \sum_{j=0}^{n_y} I(x,y)
  \end{equation}
  \item We compute the integral image using the following formula:
  \begin{equation}
   \mathcal{I}_{x,y} = \mathcal{I}_{x-1,y}+ \mathcal{I}_{x,y-1}- \mathcal{I}_{x-1,y-1}
  \end{equation}
 \end{itemize} 
}

\frame{ \frametitle{Stop Sign Detection, Integral Images}
 \begin{itemize} 
  \item Likewise, we compute an RHS integral image 
        $\mathcal{I_{+}}$ at a pixel $(x, y)$ with intensity value
        $I(x,y)$ as:
  \begin{equation}
   \mathcal{I}_{+}(x,y) = \sum_{i=N}^{n_x} \sum_{j=N}^{n_y} I(x,y)
  \end{equation}
 \end{itemize} 
}

\frame{ \frametitle{Stop Sign Detection, Integral Images}
 \begin{itemize} 
  \item After obtaining the integral image, we copy its diagonal
        into a vector $u$.
  \item We then apply a finite-difference method to the elements in
        $u$ and store it in $v$, as follows:
  \begin{equation}
   v_{n} = u_{n} - u_{n-1}
  \end{equation}
  The vector $v$ gives the LHS crosshair of the R-channel. For images
  which have stop signs, $v$ has a Gauss distribution.
 \end{itemize} 
}

\frame{ \frametitle{Stop Sign Detection, Integral Images}
 \begin{itemize} 
  \item We apply this finite-difference method for both vectors
        $u_{-}$ and $u_{+}$ to obtain $v_{-}$ and $v_{+}$.
  \item Then, we add $v_{-}$ and $v_{+}$ to obtain a vector
        $m$.
  \item Finally, we compute the standard deviation $\sigma$ of
        the vector $m$ and its centroid $c$, then apply a Gaussian
        fit to the data in $m$.
  \item Using the threshold value $\sigma_\epsilon$, we bound the
        coordinates of the box surrounding the stop sign.
 \end{itemize} 
}

\section{Warning Signs}

\frame{ \frametitle{Warning Signs}
  \begin{itemize}
    \item For our warning sign detection, we experimented with 
          SURF in combination with perspective transformations
          on out-of-plane-rotated signs. 
    \item In particular, we assume that we know the perspective
          information of an out-of-plane-rotated sign.  We apply
          a perspective transform to a model sign, then apply
          SURF to the two images, then match.
  \end{itemize} 
}

\frame{ \frametitle{Warning Signs}
  \begin{itemize}
    \item We first hand-annotated $N$ images using a MATLAB code.
    \item From this, we extracted a set of $4N$ points which
          give the vertices of the warning sign.  We then applied
          \texttt{getPerspectiveTransform()} on the points, which
          returns a perspective transform matrix \texttt{M}.
    \item We apply this perspective transform matrix to the model
          with the function \texttt{warpPerspective()}.
    \item We then run SURF on the two images, then compare their
          feature descriptors to obtain a match within a threshold
          of $t_\epsilon$.
  \end{itemize} 
}

\section{Results and Conclusion}

\frame{ \frametitle{Results}
  \begin{itemize}
    \item The following table summarizes our results:
  \end{itemize} 
  \begin{tabular}{llll}
  A & B & C & D \\
  & & &  \\
  \end{tabular}
  \begin{itemize}
    \item As is evident in the above, our accuracy for \ldots was an 
    abysmal $N\%$.
  \end{itemize} 
}

\end{document}
